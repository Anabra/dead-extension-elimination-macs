\documentclass[main.tex]{subfiles}
\begin{document}
	
	\newcommand{\mc}[3]{\multicolumn{#1}{#2}{#3}}
	\newcommand{\required}{\mc{1}{c|}{\cmark}}
	\newcommand{\unused}  {\mc{1}{c||}{\xmark}}
	
	Besides verifying the tool within an isolated environment using unit tests, we also applied it to real life Haskell packages. Namely, these packages were \pilcode{pandoc}~\cite{pandoc}, \pilcode{http-client}~\cite{http-client} and \pilcode{references}~\cite{references-bib}. \pilcode{pandoc} and \pilcode{http-client} were chosen because they are well-known, commonly used, huge Haskell packages. However, we also wanted to test our tool on a package using more type system related language extensions. For that purpose, we chose the \pilcode{references} library.
	
	While running the tests, the tool was configured with a complexity function ranking extensions based on how many other extensions they imply (Section~\ref{ranking-solutions}). Furthermore, we wanted to avoid false positive results, so all certainty levels were taken into consideration when calculating the final extension set.
	
	In tables (\ref{table:pandoc-results})--(\ref{table:references-results}), we illustrate the results of applying the tool to the packages mentioned above. In these tables only those extensions are shown whose usage statistics were impacted by the refactoring. The check marks indicate, the extension was really required, the crosses indicate the extension was enabled unnecessarily. The extensions not affected by the refactoring are aggregated in to the "other" category. During the extension elimination process, the algorithm gave \underline{no false positive results}, so every single extension the tool removed was unnecessary.
	
	\subsection{pandoc and http-client}
	
	The summarized results for \pilcode{pandoc} and \pilcode{http-client} can be seen in Tables~[\ref{table:pandoc-results},\ref{table:http-client-results}].
	
	\begin{center}
		\scalebox{0.85}
		{
			\begin{minipage}{0.58\linewidth}
				\captionof{table}{Pandoc's extension usage statistics}
				\label{table:pandoc-results}
				\begin{tcolorbox}[tab2,tabularx={l||r|r||r}]
					Extension 							& \required & \unused  & Redun.      \\
					\hline\hline
					TypeSynonymInstances    &   0       &  7       & 100\% \\\hline
					OverloadedStrings				&	 31				&  2			 & 6\%		\\\hline
					RecordWildCards         &   4       &  1       & 20\% \\\hline
					GADTs                   &   1       &  3       & 75\%	\\\hline
					FlexibleInstances       &   9       &  9       & 50\% \\\hline
					FlexibleContexts        &  15       &  4       & 21\% \\\hline
					ExplicitForAll          &   3       &  3       & 50\% \\\hline
					CPP                     &   8       &  9       & 52.9\% \\\hline
					Other										&  90				&  0 		   & 0\%  \\
					\hline\hline
					\textbf{Overall}			  & 161       & 38       & \textbf{19.2\%} \\
				\end{tcolorbox}	
			\end{minipage}
			\begin{minipage}{0.57\linewidth}
				\captionof{table}{Http-client's extension usage statistics}
				\label{table:http-client-results}
				\begin{tcolorbox}[tab2,tabularx={l||r|r||r}]
					Extension             & \required & \unused  & Redun.      \\
					\hline\hline
					TupleSections         &   1       &  1       & 50\% \\\hline
					ScopedTypeVariables   &   7       &  1       & 12,5\% \\\hline
					RecordWildCards       &   3       &  1       & 25\% \\\hline
					MultiParamTypeClasses &   0       &  2       & 100\% \\\hline
					FlexibleContexts      &   4       &  3       & 42,8\% \\\hline
					Derive[\ldots]		    &   6       &  5       & 45,5\% \\\hline
					BangPatterns          &   1       &  1       & 50\% \\\hline
					CPP                   &   9       &  7       & 43,8\% \\\hline
					Other									&  54				&  0 		   & 0\%  \\
					\hline\hline
					\textbf{Overall}			& 85        & 22       & \textbf{25.8\%} \\
				\end{tcolorbox}	
			\end{minipage}
		}
	\end{center}
	
	One of the most commonly used extensions in all Haskell packages is \ext{CPP}. We can see that the extension has a high occurrence rate in these two packages as well. However, we can also observe that it was enabled unnecessarily almost half of the time. A possible explanation for this can be that the C preprocessor is used so frequently in bigger projects, that programmers treat it more carelessly.
	
	
	Another common phenomenon we can observe in both packages is that type system related extension have relatively high redundancy compared to other extensions. For example, in \pilcode{pandoc}, we can see that \ext{TypeSynonymInstances} was enabled in seven different modules, completely unnecessarily in all of them. We can discover a similar pattern for other type system related extensions as well, such as \ext{GADTs}, \ext{FlexibleInstances} and \ext{FlexibleContexts}. Together, these four extensions are responsible for over 60\% of the total redundancy in \pilcode{pandoc}. The type system related extensions are misused in \pilcode{http-client} as well.
	
	\subsection{references}
	
	Finally, the results for the \pilcode{references} library can be seen in Table~\ref{table:references-results}.
	
	The \pilcode{references} package provides generic purpose data accessors for arbitrary types. To achieve this, the package uses numerous type system related extensions.	We chose this library to test our theory whether type system related extensions are enabled unnecessarily more often in Haskell modules. As we can see from the results, the assumption was correct, \pilcode{references} had the highest redundancy from all three packages. This redundancy mainly comes from the type system related extensions such as \ext{RankNTypes}, \ext{MultiParamTypeClasses} and \ext{FlexibleInstances}. We can also observe that the type system related extensions present in the other two libraries were mostly enabled needlessly as well.
	
	\begin{center}
		\scalebox{0.8}
		{
			\begin{minipage}{0.77\linewidth}
				\captionof{table}{References' extension usage statistics}
				\label{table:references-results}
				\begin{tcolorbox}[tab2,tabularx={l||r|r||r}]
					Extension              & Required  & Unused   & Redundancy      \\
					\hline\hline
					UndecidableInstances   &   0       &  1       & 100\% \\\hline
					TypeOperators          &   2       &  3       & 60\% \\\hline
					TypeFamilies           &   3       &  2       & 40\% \\\hline
					TupleSections          &   0       &  1       & 100\% \\\hline
					RankNTypes             &   3       &  4       & 57,1\% \\\hline
					MultiParamTypeClasses  &   3       &  4       & 57,1\% \\\hline
					LambdaCase             &   5       &  1       & 16,7\% \\\hline
					KindSignatures         &   0       &  1       & 100\% \\\hline
					FunctionalDependencies &   1       &  1       & 50\% \\\hline
					FlexibleInstances      &   5       &  3       & 37,5\% \\\hline
					FlexibleContexts       &   8       &  2       & 20\% \\\hline
					CPP                    &   0       &  1       & 100\% \\\hline
					Other									 &  40			 &  0 		  & 0\%  \\
					\hline\hline
					\textbf{Overall}   		 &  70       & 22       & \textbf{31.4\%} \\
				\end{tcolorbox}	
			\end{minipage}
		}
	\end{center}
	
	Throughout the three packages, syntactic extensions were used rarely, but mostly appropriately. A possible explanation for this can come from the nature of syntactic extensions and programming styles. Every programmer has their own coding style, which either includes using syntactic sugar introduced by certain extensions, or not. However, if the programmer does use some kind of syntax sugar, then they will always use it. Moreover, the extensions of GHC introduce syntactic sugar for very commonly used language constructs, such as \pilcode{case} and \pilcode{if} expressions, or pattern matching. As a consequence, the language extensions enabling these alternative syntactic constructs less frequently become redundant.
	
	\subsection{Limitations}
	
	\ext{CPP} and \ext{TemplateHaskell} can modify the source code at the time of prepocessing and compilation. In the case of \ext{CPP}, this modification can depend on certain compiler flags, or even on the machine we compile our program on. Regarding \ext{TemplateHaskell}, our code can be generated by arbitrarily complex meta-programs. As a result, the source code of our module can be altered in many different ways. The compiler itself will only get a single variation of the source code, which means, the tool will only be able to analyze one particular version of many possible syntax trees. As a consequence, it can only determine the minimally required extensions in one specific scenario.
	
	Currently, the tool will not remove any extensions if either \ext{CPP} or \ext{TemplateHaskell} is used in the module. We tried to partially solve this issue by prioritizing these two extensions during the elimination process. This way, if they are enabled but not used, we can remove them first, then all the other unnecessary extensions. Furthermore, we wanted to provide some information for the user even if those extensions are used, so even in that case the analysis is still performed on the module, and the language elements requiring certain extensions are highlighted.

		
\end{document}