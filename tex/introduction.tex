\documentclass[main.tex]{subfiles}
\begin{document}
	
	Haskell is a unique programming language in the sense that some of its compilers such as GHC~\cite{ghc-bib} and UHC~\cite{uhc-bib} not only implemented the most recent standard~\cite{haskell2010-bib}, but also added completely new features to the language. These features are accessible through certain language extensions~\cite{ghc-users-guide-bib}, enabled by so-called \emph{language pragmas}, declared in the first few lines of a module\footnote{Modules are the compilation units of Haskell.}. Language extensions can bring numerous new functionalities into the language. These functionalities can range from only introducing some syntactic sugar to the language to even modifying the underlying type system of the compiler. 
	
	During the software development cycle, language extensions can easily pile up in Haskell modules. A long list of language extensions can not only impair the readability of our source code, but also increase the compilation time, and reduce the portability of the code base. In many cases, a substantial amount of these extensions are unnecessarily enabled, which means they could be safely removed from the source code. Unfortunately, currently neither the compilers, nor any external tools can offer meaningful advice about extensions used in a module. Furthermore, some language extensions do not even have any syntactic evidence present in the source code, which makes determining their necessity particularly difficult. As the software evolves over time, these extensions might become redundant, but developers can hardly determine whether they are still needed or not. As a result, the number of (possibly unused) extensions will only increase over time.
	
	The main contribution of this paper is a method which can be used to automatically eliminate all unnecessary extensions from a Haskell module, and locate the program elements still requiring the remaining language extensions. Using this extra information, developers can more easily understand why a certain extension is needed in their programs. The method is implemented in the Haskell-Tools framework.
	
\end{document}