\documentclass[main.tex]{subfiles}
\begin{document}
		
		Some compilers of other languages like \pilcode{C} or \pilcode{C++} also support certain language extensions. For instance, the GNU C compiler~\cite{gnu-docs} offers several new features for the language. It also has a compiler flag that can determine which extensions are used in the program. Since most of these language extensions require using macros, or predefined functions, checking whether they are actually used can be determined at the time of preprocessing. However, for a few of these extensions like nested functions, a simple syntactical analysis is required. Furthermore, these extensions cannot be enabled or disabled by the user, they come with the compiler by default.
		
		Unlike the previously mentioned languages, Scala supports language extensions ranging from purely syntactic ones such as postfix operators to those extending the type system itself, like higher-kinded and existential types. These language features can enabled by importing the corresponding namespace. However, the programs utilizing these features will compile just fine even without the required imports. In these cases, the compiler will issue warning messages to the user that they should import corresponding feature in to their modules. This means we can easily determine the exact features used in the module by analyzing the warning messages.
		
		As for Haskell, there are only a few tools that can help programmers eliminate unused language extensions. GHC can only determine if a certain extension is needed, but is not enabled. In this case, the compilation fails, and the compiler shows an error message. Unfortunately, GHC only reports the first few missing extensions, and then it instantaneously aborts compilation. Collecting the extensions from error messages would require multiple recompilations of the module, so this is not a feasible option.
		
		There is only a single tool which made an attempt at solving the extension elimination problem. This tool is called HLint~\cite{hlint-bib}. HLint is a linter for Haskell, which can suggest possible improvements to the source code. The tool operates in a purely syntactic way. It uses \pilcode{haskell-src-exts}~\cite{haskell-src-exts} to parse the Haskell source code, then it looks for patterns in the syntax tree to identify code smells, reduce code duplication or even find unnecessary language extensions. However, since the tool only has access to syntactic information, it cannot say anything about extensions requiring semantic analysis.
	
\end{document}